\section{Les limites}

La limite d'une fonction $f$ en un point $b \in \R$ est la valeur que $f(x)$ approche lorsque $x$ se rapproche de $b$.

On note alors :
\[
\lim_{x \to b} f(x) = \ell,
\]
et on lit : “la limite de $f(x)$ lorsque $x$ s'approche de $b$ est égale à $\ell$”.

Formellement, cela signifie :
\begin{equation*}
\forall \eps > 0,\ \exists \delta > 0\ \text{tel que pour tout } x \in D_f,\ 
0 < |x - b| < \delta \implies |f(x) - \ell| < \eps.
\end{equation*}

