\section{Linéarité de la dérivation}\label{sec:9}

On utilise la définition :
\[
    f'(x) = \lim_{h \to 0} \frac{f(x+h) - f(x)}{h},
\]
lorsque cette limite existe.

\paragraph{Multiplication par une constante}

Soit $f : \R \to \R$ dérivable en $x$ et soit $\alpha \in \R$. On définit $g(x) = \alpha f(x)$. Alors :
\[
    \frac{g(x+h) - g(x)}{h}
    = \frac{\alpha f(x+h) - \alpha f(x)}{h}
    = \alpha \frac{f(x+h) - f(x)}{h}.
\]
En passant à la limite lorsque $h \to 0$, on obtient :
\[
    g'(x) = \alpha f'(x).
\]

\paragraph{Somme de fonctions}

Soient $f$ et $g$ dérivables en $x$ et soit $h(x) = f(x) + g(x)$. Alors :
\[
    \frac{h(x+h) - h(x)}{h}
\]
\[
    = \frac{f(x+h) + g(x+h) - f(x) - g(x)}{h}
\]
\[
    = \frac{f(x+h) - f(x)}{h} + \frac{g(x+h) - g(x)}{h}.
\]

En passant à la limite lorsque $h \to 0$, on obtient :
\[
    h'(x) = f'(x) + g'(x).
\]
