\section{La continuité}

\paragraph{Définition en un point.}
Soit $f : D_f \to \R$ et $c \in D_f$.  
La fonction $f$ est continue en $c$ si
\[
    \lim_{x \to c} f(x) = f(c).
\]
Ceci signifie :
\[
    \forall \eps > 0,\ \exists \delta > 0 \text{ tel que } |x-c| < \delta \implies |f(x)-f(c)| < \eps.
\]

\paragraph{Continuité sur un intervalle.}
La fonction $f$ est continue sur un intervalle ouvert $]a,b[$ si elle est continue en tout point de $]a,b[$.  
Elle est continue sur un intervalle fermé $[a,b]$ si elle est continue sur $]a,b[$, continue à droite en $a$ et continue à gauche en $b$.

\paragraph{Exemples de discontinuité.}

\paragraph{Discontinuité par saut.}
On considère
\[
f(x)=
\begin{cases}
    x+1 & \text{si } x<0,\\
    2 & \text{si } x=0,\\
    x^2 & \text{si } x>0.
\end{cases}
\]

On a

\[
    \lim_{x\to 0^-} f(x)=1,\qquad \lim_{x\to 0^+} f(x)=0,\qquad f(0)=2.
\]
Les limites à gauche et à droite sont différentes, donc la limite en $0$ n’existe pas et $f$ n’est pas continue en $0$.

\begin{figure}[H]
    \centering
    \includegraphics[width=0.3\linewidth]{images/piecewise.png}
    \caption{Discontinuité par saut en $0$.}
\end{figure}

\paragraph{Discontinuité par absence de définition.}
On considère
\[
g(x)=\frac{5+x^2}{x-3}.
\]
La fonction est définie sur $\R\setminus\{3\}$.
Elle n’est pas définie en $x=3$, donc elle ne peut pas être continue en ce point.

\begin{figure}[H]
    \centering
    \includegraphics[width=0.3\linewidth]{images/quotient.png}
    \caption{Comportement de $g$ au voisinage de $3$.}
\end{figure}

