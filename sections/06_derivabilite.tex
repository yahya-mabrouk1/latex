\section{La dérivabilité}

\paragraph{Dérivabilité en un point $x_0$} 

Une fonction $f$ est dite dérivable en $x_0$ si la limite suivante existe et est un nombre réel :
\[
    \lim_{x \to x_0} \frac{f(x)-f(x_0)}{x-x_0} = \ell \quad \text{, } \ell \in \R.
\]
Lorsque cette limite existe, $\ell$ est appelé le \textbf{nombre dérivé} de $f$ en $x_0$ et se note $f'(x_0)$, lu ``f prime de $x_0$''.

\paragraph{Dérivabilité sur un intervalle} 

Une fonction $f$ est dérivable sur un intervalle $I = [a,b]$ si elle est dérivable en tout point $x \in I$. 
Dans ce cas, on définit la fonction dérivée de $f$, notée $f'$, qui à tout $x \in I$ associe son nombre dérivé :
\[
    f' : I \longrightarrow \R, \quad x \longmapsto f'(x).
\]
