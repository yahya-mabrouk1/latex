\section{Méthode de Newton}\label{sec:11}

\paragraph{Principe}

On cherche une solution de $f(x) = 0$.  
À partir d’une approximation initiale $x_0$, on définit la suite :
\[
    x_{n+1} = x_n - \frac{f(x_n)}{f'(x_n)}.
\]

\paragraph{Algorithme (Python)}

\begin{lstlisting}
def newton(f, df, x0, N):
    """
    Applique la methode de Newton
    pour approcher une racine de la fonction f.

    Parametres
    ----------
    f : fonction
        Fonction dont on cherche une racine (f(x) = 0).
    df : fonction
        Derivee de la fonction f.
    x0 : float
        Approximation initiale de la racine.
    N : int
        Nombre d iterations de la methode.

    Retour
    ------
    float
        Approximation de la racine apres N iterations.
    """
    x = x0
    for _ in range(N):
        x = x - f(x)/df(x)
    return x
\end{lstlisting}

\paragraph{Exemple}

On cherche une approximation de $\sqrt{2}$ comme racine de $f(x) = x^2 - 2$.
On a $f'(x) = 2x$ et on choisit $x_0 = 1$.

\[
    x_1 = 1 - \frac{1^2 - 2}{2 \cdot 1} = 1.5,
    \qquad
    x_2 = 1.5 - \frac{1.5^2 - 2}{2 \cdot 1.5} = 1.416\overline{6}.
\]

On obtient rapidement $\sqrt{2} \approx 1.4142$.
