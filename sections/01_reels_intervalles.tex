\section{L'intervalle $\R$}

En mathématiques, un nombre réel est un nombre qui appartient à l’ensemble $\mathbb{R}$.  
Les nombres réels permettent de mesurer des quantités unidimensionnelles comme une longueur, une durée ou un montant.

Le symbole $\R$ désigne l’ensemble de tous les nombres réels. On peut le représenter par l’intervalle $]-\infty, +\infty[$.

Un intervalle de $\R$ est un sous-ensemble $I \subset \R$ tel que pour tous $a,b \in I$ avec $a \le b$, tout réel $x$ vérifiant $a \le x \le b$ appartient aussi à $I$.
