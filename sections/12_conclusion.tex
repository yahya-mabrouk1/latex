\section{Conclusion}

Les sections précédentes ont construit progressivement les outils fondamentaux du calcul différentiel.

Dans \hyperref[sec:8]{la section 8}, on établit rigoureusement les dérivées des fonctions constantes, affines et quadratiques à partir de la définition par limite.  
Dans \hyperref[sec:9]{la section 9}, on montre que la dérivation est un opérateur linéaire, ce qui permet de dériver facilement des combinaisons de fonctions.  
Dans \hyperref[sec:10]{la section 10}, on généralise ces résultats à toutes les puissances entières par récurrence, ce qui donne accès à une large classe de fonctions polynomiales.

Ces résultats fournissent le cadre théorique nécessaire à la méthode de Newton présentée dans \hyperref[sec:11]{la section 11}. La méthode repose explicitement sur l’existence et le calcul de la dérivée, et sa convergence rapide est une conséquence directe des propriétés locales des fonctions dérivables.

On voit ainsi comment une notion analytique abstraite (la dérivée définie par une limite) conduit à un algorithme numérique efficace pour le calcul approché des zéros de fonctions.

\section*{Références}

\begin{itemize}
    \item Wikipédia, \emph{Limite d’une fonction}.
    \href{https://en.wikipedia.org/wiki/Limit_(mathematics)}{Lien}

    \item Wikipédia, \emph{Fonction réelle d’une variable réelle}.
    \href{https://fr.wikipedia.org/wiki/Fonction_r%C3%A9elle_d'une_variable_r%C3%A9elle}{Lien}

    \item Wikipédia, \emph{Continuité}.
    \href{https://fr.wikipedia.org/wiki/Continuit%C3%A9_(math%C3%A9matiques)}{Lien}

    \item Wikipédia, \emph{Dérivée}.
    \href{https://en.wikipedia.org/wiki/Derivative}{Lien}

    \item APEX Calculus (LibreTexts), 
    \emph{Epsilon--Delta Definition of a Limit}.
    \href{https://math.libretexts.org/Bookshelves/Calculus/Calculus_3e_(Apex)/01%3A_Limits/1.02%3A_Epsilon-Delta_Definition_of_a_Limit}{Lien}

    \item Real Python, \emph{Documenting Python Code}.
    \href{https://realpython.com/documenting-python-code/#documenting-your-python-code-base-using-docstrings}{Lien}

    \item Wikipédia, \emph{Newton's Method}.
    \href{https://en.wikipedia.org/wiki/Newton's_method}{Lien}

    \item D. Homola, \emph{Newton’s Method in Python}.
    \href{https://danielhomola.com/learning/newtons-method-with-10-lines-of-python/}{Lien}
\end{itemize}
